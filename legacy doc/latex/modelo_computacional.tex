%modelo computacional

\section{Modelo computacional}

El problema es escencialmente numérico y por lo tanto
computacional. Al ser un problema no resoluble analíticamente en su
forma más general, abre la pósibilidad del uso de técnicas numéricos y
las estructuras de datos que la hacen atractiva desde un punto de
vista algorítmico.

desconosido que se desea estimar, entonces el problema es diferente y
puede llegar a ser no trivial. La solución puede no existir, de
existir puede ser no única y de ser única puede no ser estable. Para
demostrar la existencia, unicidad y estabilidad de la solución son
partes del desafio de la tesis.

Este tipo de problemas son denominados problemas inversos los cuales
muchos de ellos pueden ser denominados como problemas inversos mal
puestos y que requieren de técnicas de regularización y agregar.
Determinar si es un problema paralelizable, el tratamiento en terminos
computacionales y cuales son las estructuras de datos necesarias para
manipular correcta y/o eficientemente el problema no son cuestiones de
respuesta inmediata. En la medida que la investigación vaya
evolucionando irán apareciendo problemas que van a ser resueltos
primero con un prototipo suboptimo y luego su implementación definitiva.

Estas preguntas no son de respuesta inmediata y dependen del problema
que se esté tratando, pero está a la vista que la abundancia de
hardware permite reflexionar sobre las posibilidades de optimización
del problema en la media que ya se tenga una solucón subóptima, por
esta razón considerará en la medida que se deba de optimizar la
solución existente.