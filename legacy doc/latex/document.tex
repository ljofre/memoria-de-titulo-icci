\documentclass{article}

\usepackage{amssymb}
\usepackage{amsfonts}

\usepackage[utf8]{inputenc}

%%para usar dot2tex y generar grafos y diagramas avanzados dentro de LaTeX
\usepackage{tikz}
\usetikzlibrary{arrows}
\usetikzlibrary{backgrounds}
\usetikzlibrary{fit}
\usetikzlibrary{calc}
\usepackage{dot2texi}


%%nuevos comandos especiales para este trabajo
\newcommand{\dpartial}[2]{\frac{\partial #1}{\partial #2}}
\newcommand{\Real}{\mathbb{R}}
\newcommand{\nodes}{$\left\{ \left( u_i,v_i \right) \right\}_i$}
\newcommand{\tupla}[1]{\left(#1_1,#1_2,\cdots,#1_k\right)}

\title{Métodos paralelos para la resolución de un problema inverso, aplicacion sobre un problema de gravimetria}


% las herramientas usadas deben ir en el anexo

\author{Leonardo Andrés Jofré Flor}
\begin{document}
\maketitle
%% un prefacio es, en literatura, un texto de introducción y de
%% presentacion, ubicado al inicio del libro.

\section{Prefacio}

% diagrama de la idea:
% (1) el concepto errado de computación cuando se confunde con la informática
% (2) la computación como técnica al servicio de las matemáticas, las matemáticas al servicio de la ingeniería.
% (3) La computación tiene sus inicios en el cálculo numérico.
% (4) Algebra lineal herramienta del computo para resolver problemas de algebra lineal


%terminar esta idea
Esto ha resultado en que resolver problemas de la física computacionalmente si es que nos
podemos referir a la computación solo como cálculo, es
una estrategia que ha ido tomando fuerza dada la
posibilidad de obtener soluciones cada vez mas eficientemente a problemas cada vez más grandes. Esto se
produce por evolución de la capacidad de computo de los procesadores,
y estudio de las estructuras de datos 
solución en tiempos de cómputo cada vez de menor orden en espacio para computo y tiempo de cálculo, también métodos numéricos resuelven problemas clásicos del álgebra lineal que aparecen al discretizar sistemas continuos que no se pueden resolver analiticamente.

%Acerca de los algoritmos y el hardware




%acerca de el poder de cómputo

%explicar que el error conceptual generado
Esto de la computación, que se refiere a la capacidad a la capacidad de uso
intensivo de computadores, concepto ligeramente impromipio, dado que
el instrumento no define la técnica 

%ingresar este texto
estudiando modelos algorítmicos de 
fenómenos físicos, permite resolver problemas que no son tratables 
analíticamente, o como se suele referir, requieren de métodos numérico.

Luego, la computación permite un interesante juego teórico: Si se tiene un
modelo detallado de la realidad que se aspira a comprender, se puede puede
simular el comportamiento en la memoria de un computador para lograr medir
cantidades relevantes, y en la medida de que la computación y
algoritmos avanzan nos permiten generar herramientas automatizadas.


El presente trabajo tiene nace en el contexto de crear las
herramientas necesarias para la resolución más general de una linea de
investigación de gravimetría para el laboratorio de modelamiento
matemático para geomecánica (MMGEO) que tenía como objetivo estimar el
cave-back a partir de las mediciones de gravedad (verticales) en
terreno. Dada las necesidades de considerar geometrías más generales y
además considerar la información de la producción de material de la
mina es necesario hacer consideración de una nuevo método numérico que
se base en otros supuestos %cuales supuestos

La forma en que se desarrolló este trabajo depende fuertemente de la
evolución de cada uno de los problemas puntuales que he tenido la
explorados, buscando metodologías para la resolución de problemas,
sirviendo de apoyo para otros matemáticos o creando nuevas
implementaciones para los papers que instuiamos en que se podrán
encontrar las soluciones para los subproblemas.

Esto quiere decir que el desarrollo, aunque consideraba de mucho
material y ya mucha implementación para la fecha en que se inicio esta
tesis, gran parte del trabajo fue darle una secuencialidad, ordenando,
revisando y desechando los multiples borradores en papel y papers
sueltos.

Mucho tiene de resumen del trabajo ya hecho, pero así mismo ese
resumen recoje los argumentos necesarios para justificar los futuros
trabajos que pueden ser posibles o en los que ya se están trabajando.

Es posible que cada uno de los capítulos de este trabajo tiendan a
parecer un trabajo independiente, eso es totalmente intencional, las
razones son que cada uno de los términos de este trabajo se han
desarrollado de forma independiente pero sin una conexión entre estas,
cada una ha intentado trabajar con la física de forma
independiente. Esto más que un defecto puede considerarse una
oportunidad, dado que, un trabajo importante seráa la interelación de
distintos modelos para obtener las generalizaciones de un modelo
unificado.




%% en la introducción van los objetivos generales y los objetivos
%% específicos del proyecto en particular.

\section{Introducción}

% informacion adicional, restricciones
% informacion de la producción
% busqueda de un volumen

% el modelo matemát


ico implica un aumento en la matrz por invertir en funcion
% de la cantidad de restricciones y cantidad de nodos con que se quere estimar el asunto.
% el problema de discretización implica agregar un error adicional

% un problema computacional
% estructura de datos, hardware.
% que es la gravimetría

El concepto de Problema inverso, como su nombre lo indica, el camino contrario al problema directo
Todo efecto proviene de una causa, ( ... ) un problema se puede denominar como directo si conocemos las leyes que rigen un fenómeno y las causa de un fenómeno con lo cual es muy directo computar algunos parámetros que caracterizan a los efectos. El problema inverso es distinto está inscritos en una rama relativamente reciente de las matemáticas.  En el problema inverso se intenta deducir las causas (resumidas en los parámetros de un modelo) a partir de los datos observados. Dado que para estimar los parámetros del modelo, nos basamos en métodos que dependen de los datos observados hace que nos enfrentemos a un conjunto de preguntas que son de importancia resolver y que implican respuestas en una formulación matemática, lo que implicaría la posibilidad de un diseño de computo suficiente para obtener.

Además, para tratar computacionalmente el problema, los modelos deben ser discretizados y esto introduce errores adicionales.




La gravimetría es un proceso importante en la búsqueda de depósitos minerales 

El Hace referencia a la computación de alto rendimiento para resolver un
problema físico que nace fundamentalmente de la minería.

El presente trabajo presenta algunos avances que hacen referencia a la
reconsideración de dominios más generales lo que impacta directamente
en los métodos computacionales y abre camino para el uso más eficiente
del hardware mediante computación paralela.

De esto se deduce que el modelo posiblemente mejorable en los siguientes aspectos: 
\begin{enumerate}
\item Considerar el volumen como una nueva restricción.
\item Mejorar la velocidad de convergencia del método numérico.
\item Aspectos de visualización de los resultados.
\item considerar distribuciones más generales para la estimación de la
  forma del volumen.
\item Reestimar el parametro $\alpha$ para la regularización de
  Tychonoff para el problema mal puesto.
\end{enumerate}


\subsection{objetivos generales}
Exponer los métodos computacionlaes para resolver problemas científicos que
requieren de alto rendimiento.

Formular un método general para estimar volúmenes que generen campos
escalares cocidos. Crear un nuevo modelo computacional de problema
inverso para poder reconstruir el volumen buscado.

Comparar los resultados anteriores ...


\subsection{objetivos específicos}

crear una aplicaciones que haga uso de un cluster de alto desempeño mediante
computación paralela me paso de mensajes para la resolución de un problema de
gravimetría.

Por otra parte, exportar librerías de mallado que ya se encuntran en
C++ a python, como lo son TetGen, meiante swig y específicamente
mediante una biblioteca llamada instant, esta parte se hará al final
del desarrollo ya que pertenece a una parte de optimización de lo ya
existente.

%visualizacion de los objetivos mediante el siguiente diagrama



\section{Planteamiento del problema}

Sea $p \in \mathbb{R}^3$ una particula que genera un campo escalar en
la coordenada $x$ de la forma $F(x-u)$ donde $u$ son las coordenadas de $p$.

Si consideramos un volumen $V$ compuesta por la superposición de estas
partículas, el campo escalar generado por este volumen estará
determinado por la siguiente integral: $$\psi(x)=\int_V
\rho(u)F(x-u)dV$$ donde $\rho$ es una función de
densidad para la coordenada $u$.

El cálculo del campo escalar $\psi$ en todo el medio es trivial, dado
que es una integral sobre un volumen ya conocido y el cálculo desde un
punto de vista computacional y matemático carece de atractivo para la
mayoría de los casos.

Podríamos decir entonces que el volumen $V$ es la causa del campo
escalar $\psi$ y este campo se puede obtener conociendo $F$ y $\rho$

Por otro lado, los problemas de ingeniería en los cuales se desea
estimar las causas de un fenómeno siempre tiene una cantidad limitada
de información, y que además pueden contener errores.Si se conocen
para un conjunto finito de puntos en el espacio $(x_i,\psi(x_i))$ los
campos escalares producidos se desea recuperar la forma del volumen
$V$ que genera esos campos específicamente.


\section{Propuesta de solución}

Para este se hay que recuperar la frontera como incognita que produce
esos vectores conocidos, para ello se usará una estructura de datos de
malla generada mediante una triangulación de Delaunay en la cual
mediante un método numérico de Newton-Raphson se iterará cada nodo de
la frontera hasta converger a la solución buscada.

Se propone una solución mediante un método numérico que toma una malla
inicial $\Omega$ que representa el volumen del sólido y que
iterativamente converja a una malla que busca minimizar la siguiente
integral.

$$I(\Omega) = \sum_i(\psi(x_i) - \int_\Omega F(x_i-u)dV)^2$$

Consideremos que existe una función $G:\mathbb{R}^2\to \mathbb{R}$ que
cumple siguiente propiedad

$$\nabla \cdot G(u) =F(x_i-u)$$

entonces, gracias al teorema de la divergencia:

$$I(\Omega) = \sum_i(\psi(x_i) - \int_{\partial \Omega} \vec{G} \cdot \vec{n} dS)^2$$ donde $\vec{n}$ es el vector
normal a la superficie volumen buscado.

\subsection{Modelo discretizado}
Dado que el minimizador de $I$ no se puede obtener de forma analítica
en su forma más general se hará uso de un método numérico
para encontrar las coordenadas nodales para la discretización de la
frontera $\Omega$.

Si consideramos $\tilde \Omega$ una discretización tetrahedrizada del
dominio original $\Omega$ donde se cumple la siguientes propiedades:

$$\bigcup_{T_i \in \tilde{\Omega}} T_i =  \tilde{\Omega}$$
como es natural y que además que cada una de sus componentes sean
disyuntas, esto quiere decir que 

$$\bigcap_{T_i \in \tilde{\Omega}} T_i = \varnothing$$

por lo que la integral queda definida de la siguiente forma 

$$I(\Omega) = \sum_i(\psi(x_i) - \sum_{u_j \in \partial \tilde
  \Omega}\vec{G}_k \cdot\vec{n}_k \Delta
S_k)^2$$

En donde $u_j \in \partial \tilde \Omega$ significa que la suma se
hace sobre todos los nodos de la frontera del volumen buscado y
$\Delta S_k$ corresponde a un area asocidada que se puede ontener
mediante la siguiente formula.

$$\Delta S_k = \sum Poli_k$$ en donde $\sum_k Poli_k$ es el area de la
superficie generada por el polígono formado por todos los centroides
de cada uno de los triangulos que contienen a $u_k$ dado el mallado
específico. Para esta cuenta se puede usar el ortocentro, el
baricentro o el incentro siendo los tres buenas aproximaciones, se
seleccionará entonces la que hagan que el cálculo sea más simple.

Una caracteristica de esta medida es que se debe cumplir que

$$\lim_{N\to\infty} \sum \Delta S_k = \oint dS = A $$ en donde $A$ es el
area de la superficie para el caso continuo.

En el caso elemental, la solucion del problema es la superficie que minimize el funcional $I$ pero en el caso mas general existen una serie de restricciones que se deben de tomar en consideracion y son especificamente de cinco tipos.

1.- El volumen encerrado es conocido
2.- La frontera de la superficie es conocida en un conjunto finito de puntos
3.- La frontera de la superficie es conocida es un conjunto infinito de puntos (un segmento)
4.- la superficie esta acotada de alguna manera por otra superficie.
5.- existen segmentos de recta que estan estrictamente fuera del volumen inscrito en la superficie.



\subsection{Rediscretización}
En la medida que la frontera de discretizada de $\Omega$ vaya
actualizandose hasta converger en el minimizador de $I$ entonces la
calidad de la triangulación de la frontera va disminuyendo por lo que
es necesario rediscretizar el dominio para conservar las propiedades
óptimas dadas por la triangulación de Delaunay.

\subsection{Restricciones y regularización}
Al ser un problema del tipo inverso que admite una inestabilidad
natural, dado que los datos son por mucho inferiores a los grados de
libertad que se desean estimar, nos vemos en la necesidad de agregar
restricciones que hagan robustecer los resultados y aumentar la
velocidad de convergencia. Además se buscará usar conocimiento a
priori que sirva como restricción: Un ejemplo es que es conocida el
area de la superficie o el volumen las cuales serían restricciones de
igualdad y la segunda es que el volumen debe respetar ciertas cotas
superiores o inferiores las cuales son  restricciones de
desigualdad. Para manejar ese tiempo de restricciones se hará uso del
teorema de karush Khun Tacker que es una generalización de los
multiplicadores de Lagrange para este objetivo.

\subsection{Método de Newton-Raphson paralelo}
Finalmente, después de todo el procedimiento anterior se buscará la
solución en paralelo del método numérico para encontrar la solución,
para esto se usará la librería Petsc dado que para el método de
Newton-Raphson se requiere calcular numéricamente la inversa del
jacobiando, que es en si una matriz dispersa, esta biblioteca tiene
como objetivo ese tipo específico de problemas.

Consideremos que se tiene un sistema de la forma $$F(x)=0$$ en donde $x
= \tupla{x}$, para resolver el sistema se tiene entonces que usar
algún método iterativo eficiente, como lo puede ser Newton-Raphson,
así de esta manera, si calculamos el jacobiano $J$, el algoritmo
dependiendo de la condición inicial $x_0$ queda de la siguiente
manera.

$x_{n+1}=x_n+J^{-1}(x_n)*F(x_n)$

en donde se sabe que $J$ es una matriz dispersa

el valor de $J^{-1}(x_n)*F(x_n)$ es muy complejo de obtener mediante
métodos exactos. El valor de esta multiplicación es equivalente a la
solución del sistema 

$J(x_n)f=F(x_n)$

\subsection{Subespacio de Krilov}
El espacio de Krilov es muy importante para la resolucion de sistemas de ecuaciones en donde hay que invertir una matriz dispersa, se busca la solucion del sistema dentro de ese espacio.
El subespacio de Krilov es un sobespacio que se construye de la siguiente manera

$K_r(A,v)= span{A^0v, ..., A^{r-1} v}$

El $n-$simo espacio de Krilov , en donde $A$ es una matriz invertir 

\subsection{Uso de Generalized minimal residual method (gmres)}

Gmres aproxima la solución la solución $Ax=b$ para el vector $x_n \in K_n$ que minimiza la norma Euclidea del residuo $r_n = Ax_n-b$. 

Es un método iterativo para la solución de sistemas de ecuaciones lineales no simétricos. El método aproxima 

\subsubsection{Uso de gmres para la inversión en paralelo}





\section{Aplicaciones}

El problema que se trata de solucionar es el caso más general de la
busqueda de un medio que genera un campo escalar. Si el campo está
asociado al gravitacional entonces se trataría de un problema de
gravimetría en donde las diferencias de gravedad entre dos instantes
representan el campo gravitacional producido por un volumen
desprendido en una caverna, por otro lado, si el campo es un campo
magnético entonces se trata de un problema de magnetometría, en donde
un mineral con una densidad $\rho$ genera alteraciones el un campo
magnético, ambas técnicas requeridas en minería.

%modelo computacional

\section{Modelo computacional}

El problema es escencialmente numérico y por lo tanto
computacional. Al ser un problema no resoluble analíticamente en su
forma más general, abre la pósibilidad del uso de técnicas numéricos y
las estructuras de datos que la hacen atractiva desde un punto de
vista algorítmico.

desconosido que se desea estimar, entonces el problema es diferente y
puede llegar a ser no trivial. La solución puede no existir, de
existir puede ser no única y de ser única puede no ser estable. Para
demostrar la existencia, unicidad y estabilidad de la solución son
partes del desafio de la tesis.

Este tipo de problemas son denominados problemas inversos los cuales
muchos de ellos pueden ser denominados como problemas inversos mal
puestos y que requieren de técnicas de regularización y agregar.
Determinar si es un problema paralelizable, el tratamiento en terminos
computacionales y cuales son las estructuras de datos necesarias para
manipular correcta y/o eficientemente el problema no son cuestiones de
respuesta inmediata. En la medida que la investigación vaya
evolucionando irán apareciendo problemas que van a ser resueltos
primero con un prototipo suboptimo y luego su implementación definitiva.

Estas preguntas no son de respuesta inmediata y dependen del problema
que se esté tratando, pero está a la vista que la abundancia de
hardware permite reflexionar sobre las posibilidades de optimización
del problema en la media que ya se tenga una solucón subóptima, por
esta razón considerará en la medida que se deba de optimizar la
solución existente.


% los capítulos restantes contienen el marco teórico del problema

\section{Optimización no lineal}
El presente capítulo tiene como objetivo presentar el teorema
Khun-Tucker para optimización no lineal.  Este teorema es una
generalización a los multiplocadores de Lagrange por lo que nos
permite resolver problemas de optimización no lineal mediante
restricciones de desigualdad, además las condiciones de Kuhn-Tucker da
las condiciones necesarias para la existencia de solución de un
programa de optimización no lineal. Además se identifican los puntos
en el cual el cómputo se puede escalar horizontalmente mediante la
técnica de computación paralela.



\section{Problemas inversos}

Uno se los más famosos problemas invesos es el siguiente:¿Podemos
escuchar la forma de de un tambor?, esto quiere decir: ¿Es posible
encontrar una única forma para un tambor basándonos en el sonido que
este emite?, el problema directo es verdaderamente simple, dado que,
conocida la forma, podemos deducir el sonido que emite y su solución
es conocida ya desde hace mucho tiempo, pero para la solución invesa
la respuesta es que no, dado que dos tambores distintos pueden emitir
el mismo sonido, esto quiere decir que si escuchamos un sonido no
podríamos distinguir cual es la forma del que lo omite.

Las teorías físicas nos permiten hacer predicciones: Dada una
descripción completa de un sistema físico nosotros podemos predecir el
resultado de algunos parámetros. El problema de predecir el resultado
de las medidas es llamado \textit{problema de modelación o de
  modelamiento}. El \textit{problema inverso} consiste en que, usando
los resultados actuales inferir los valores de los parámetros que
caracterizan el sistema.

Muchos sistemas físicos pueden ser descritos usando un modelo lineal
de la forma:

$$AX = Y$$ en donde $X$ e $Y$ se consideran espacios de Hilbert.

%todo: dibujo de caja para la causa y el efecto

\subsection{Regularización}

Método de regularización de Tichonoff

La idea básica de la regularización de Tichonoff está relacionada con
la minimización del funcional cuadrático:

$$\Phi_\mu(f,g) = \| Af-g\|^2 + \mu^2 \| f\|^2$$

El minimizador se puede obtener de forma analítica de la siguiente
manera.

$$X = \left(A^*A+\mu^2 I \right)A^*Y$$

alguna bibliografía definen $\mu = \sigma_Y/\sigma_X$ como el
parámetro óptimo.



\subsection{Regularización de Tichonoff paralelo  para problemas de optimización no lineal}

La regularizacion de Tijonov se usa habitualmente para obtener soluciones relativamente suaves, dado que penaliza la norma de la solucion lo que penaliza las gradientes grandes.

Para buscar el óptimo para el siguiente problema de optimización con
restricciones

$f(x)=0$ con $g_i(x) < 0 y h_i=0$ Al generar un sistema no lineal
iterativamente se converge a la solución.

Para resolver un problema no lineal, pero que al mismo tiempo conserve cierta suavidad en la solucion se usara la siguiente tecnica.




% a modo de ejemplo se demuestra el uso en dos dimensiones

\section{Ejemplo bidimensional}

Consideremos la curva cerrada simple $C$ en el plano XY que encierra a
un área $\Omega$ y $F(x,y)=\frac{y}{\left(x^2+y^2\right)^{\frac{3}{2}}}$ un campo escalar
genereado por una unidad diferencial $dA$ entonces el campo escalar en el
punto $(x,y)$ generado por la totalidad del área encerrada por la
curva cerrada simple $C$ es
$$\psi(x,y) = \int_\Omega F(u-x,v-y)dudv$$

$$\psi(x,y) = \int_\Omega \frac{1}{(x-u)^2+(y-v)^2}dudv$$


Aplicando el teorema de Green
$$\psi(x,y) = \int_{\partial \Omega} Pdu+Qdv$$

en donde se cumple que
$\frac{\partial Q}{\partial u}-\frac{\partial P}{\partial
  v}=\frac{1}{(x-u)^2+(y-v)^2}$. Si
  consideramos que $P$ es constante en términos de $v$ entonces
  $\frac{\partial P}{\partial v}=0$, luego:

$$\frac{\partial Q}{\partial u}=\frac{1}{(u-x)^2+(v-y)^2}$$
luego $Q(u,v)=\int \frac{1}{(u-x)^2+(v-y)^2}du+h(v) $

Como $C$ es una curva cerrada simple, entonces $\int h(v)dv=0$,
finalmente, el campo escalar en todo queda definido en todo el espacio
como la siguiente función.

$$\psi(x,y)=\oint_{\partial \Omega} Q(u-x,v-y)dv$$

Si discretizamos la integral
$$\tilde \psi(x,y)=\sum_i Q(u_i-x,v_i-y)(v_{i}-v_{i-1})$$

	finalmente, se desean buscar nodos \nodes  que minimice el
error cuadrático

$$I(\Omega) = \sum_k ( \psi_k -\tilde \psi(x_k,y_k))^2$$
Se puede demostrar que es equivalente a buscar los nodos 
 
\end{document}