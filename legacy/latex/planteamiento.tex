
\section{Planteamiento del problema}

Sea $p \in \mathbb{R}^3$ una particula que genera un campo escalar en
la coordenada $x$ de la forma $F(x-u)$ donde $u$ son las coordenadas de $p$.

Si consideramos un volumen $V$ compuesta por la superposición de estas
partículas, el campo escalar generado por este volumen estará
determinado por la siguiente integral: $$\psi(x)=\int_V
\rho(u)F(x-u)dV$$ donde $\rho$ es una función de
densidad para la coordenada $u$.

El cálculo del campo escalar $\psi$ en todo el medio es trivial, dado
que es una integral sobre un volumen ya conocido y el cálculo desde un
punto de vista computacional y matemático carece de atractivo para la
mayoría de los casos.

Podríamos decir entonces que el volumen $V$ es la causa del campo
escalar $\psi$ y este campo se puede obtener conociendo $F$ y $\rho$

Por otro lado, los problemas de ingeniería en los cuales se desea
estimar las causas de un fenómeno siempre tiene una cantidad limitada
de información, y que además pueden contener errores.Si se conocen
para un conjunto finito de puntos en el espacio $(x_i,\psi(x_i))$ los
campos escalares producidos se desea recuperar la forma del volumen
$V$ que genera esos campos específicamente.


\section{Propuesta de solución}

Para este se hay que recuperar la frontera como incognita que produce
esos vectores conocidos, para ello se usará una estructura de datos de
malla generada mediante una triangulación de Delaunay en la cual
mediante un método numérico de Newton-Raphson se iterará cada nodo de
la frontera hasta converger a la solución buscada.

Se propone una solución mediante un método numérico que toma una malla
inicial $\Omega$ que representa el volumen del sólido y que
iterativamente converja a una malla que busca minimizar la siguiente
integral.

$$I(\Omega) = \sum_i(\psi(x_i) - \int_\Omega F(x_i-u)dV)^2$$

Consideremos que existe una función $G:\mathbb{R}^2\to \mathbb{R}$ que
cumple siguiente propiedad

$$\nabla \cdot G(u) =F(x_i-u)$$

entonces, gracias al teorema de la divergencia:

$$I(\Omega) = \sum_i(\psi(x_i) - \int_{\partial \Omega} \vec{G} \cdot \vec{n} dS)^2$$ donde $\vec{n}$ es el vector
normal a la superficie volumen buscado.

\subsection{Modelo discretizado}
Dado que el minimizador de $I$ no se puede obtener de forma analítica
en su forma más general se hará uso de un método numérico
para encontrar las coordenadas nodales para la discretización de la
frontera $\Omega$.

Si consideramos $\tilde \Omega$ una discretización tetrahedrizada del
dominio original $\Omega$ donde se cumple la siguientes propiedades:

$$\bigcup_{T_i \in \tilde{\Omega}} T_i =  \tilde{\Omega}$$
como es natural y que además que cada una de sus componentes sean
disyuntas, esto quiere decir que 

$$\bigcap_{T_i \in \tilde{\Omega}} T_i = \varnothing$$

por lo que la integral queda definida de la siguiente forma 

$$I(\Omega) = \sum_i(\psi(x_i) - \sum_{u_j \in \partial \tilde
  \Omega}\vec{G}_k \cdot\vec{n}_k \Delta
S_k)^2$$

En donde $u_j \in \partial \tilde \Omega$ significa que la suma se
hace sobre todos los nodos de la frontera del volumen buscado y
$\Delta S_k$ corresponde a un area asocidada que se puede ontener
mediante la siguiente formula.

$$\Delta S_k = \sum Poli_k$$ en donde $\sum_k Poli_k$ es el area de la
superficie generada por el polígono formado por todos los centroides
de cada uno de los triangulos que contienen a $u_k$ dado el mallado
específico. Para esta cuenta se puede usar el ortocentro, el
baricentro o el incentro siendo los tres buenas aproximaciones, se
seleccionará entonces la que hagan que el cálculo sea más simple.

Una caracteristica de esta medida es que se debe cumplir que

$$\lim_{N\to\infty} \sum \Delta S_k = \oint dS = A $$ en donde $A$ es el
area de la superficie para el caso continuo.

En el caso elemental, la solucion del problema es la superficie que minimize el funcional $I$ pero en el caso mas general existen una serie de restricciones que se deben de tomar en consideracion y son especificamente de cinco tipos.

1.- El volumen encerrado es conocido
2.- La frontera de la superficie es conocida en un conjunto finito de puntos
3.- La frontera de la superficie es conocida es un conjunto infinito de puntos (un segmento)
4.- la superficie esta acotada de alguna manera por otra superficie.
5.- existen segmentos de recta que estan estrictamente fuera del volumen inscrito en la superficie.



\subsection{Rediscretización}
En la medida que la frontera de discretizada de $\Omega$ vaya
actualizandose hasta converger en el minimizador de $I$ entonces la
calidad de la triangulación de la frontera va disminuyendo por lo que
es necesario rediscretizar el dominio para conservar las propiedades
óptimas dadas por la triangulación de Delaunay.

\subsection{Restricciones y regularización}
Al ser un problema del tipo inverso que admite una inestabilidad
natural, dado que los datos son por mucho inferiores a los grados de
libertad que se desean estimar, nos vemos en la necesidad de agregar
restricciones que hagan robustecer los resultados y aumentar la
velocidad de convergencia. Además se buscará usar conocimiento a
priori que sirva como restricción: Un ejemplo es que es conocida el
area de la superficie o el volumen las cuales serían restricciones de
igualdad y la segunda es que el volumen debe respetar ciertas cotas
superiores o inferiores las cuales son  restricciones de
desigualdad. Para manejar ese tiempo de restricciones se hará uso del
teorema de karush Khun Tacker que es una generalización de los
multiplicadores de Lagrange para este objetivo.

\subsection{Método de Newton-Raphson paralelo}
Finalmente, después de todo el procedimiento anterior se buscará la
solución en paralelo del método numérico para encontrar la solución,
para esto se usará la librería Petsc dado que para el método de
Newton-Raphson se requiere calcular numéricamente la inversa del
jacobiando, que es en si una matriz dispersa, esta biblioteca tiene
como objetivo ese tipo específico de problemas.

Consideremos que se tiene un sistema de la forma $$F(x)=0$$ en donde $x
= \tupla{x}$, para resolver el sistema se tiene entonces que usar
algún método iterativo eficiente, como lo puede ser Newton-Raphson,
así de esta manera, si calculamos el jacobiano $J$, el algoritmo
dependiendo de la condición inicial $x_0$ queda de la siguiente
manera.

$x_{n+1}=x_n+J^{-1}(x_n)*F(x_n)$

en donde se sabe que $J$ es una matriz dispersa

el valor de $J^{-1}(x_n)*F(x_n)$ es muy complejo de obtener mediante
métodos exactos. El valor de esta multiplicación es equivalente a la
solución del sistema 

$J(x_n)f=F(x_n)$

\subsection{Subespacio de Krilov}
El espacio de Krilov es muy importante para la resolucion de sistemas de ecuaciones en donde hay que invertir una matriz dispersa, se busca la solucion del sistema dentro de ese espacio.
El subespacio de Krilov es un sobespacio que se construye de la siguiente manera

$K_r(A,v)= span{A^0v, ..., A^{r-1} v}$

El $n-$simo espacio de Krilov , en donde $A$ es una matriz invertir 

\subsection{Uso de Generalized minimal residual method (gmres)}

Gmres aproxima la solución la solución $Ax=b$ para el vector $x_n \in K_n$ que minimiza la norma Euclidea del residuo $r_n = Ax_n-b$. 

Es un método iterativo para la solución de sistemas de ecuaciones lineales no simétricos. El método aproxima 

\subsubsection{Uso de gmres para la inversión en paralelo}





\section{Aplicaciones}

El problema que se trata de solucionar es el caso más general de la
busqueda de un medio que genera un campo escalar. Si el campo está
asociado al gravitacional entonces se trataría de un problema de
gravimetría en donde las diferencias de gravedad entre dos instantes
representan el campo gravitacional producido por un volumen
desprendido en una caverna, por otro lado, si el campo es un campo
magnético entonces se trata de un problema de magnetometría, en donde
un mineral con una densidad $\rho$ genera alteraciones el un campo
magnético, ambas técnicas requeridas en minería.
