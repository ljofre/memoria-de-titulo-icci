
\section{Problemas inversos}

Uno se los más famosos problemas invesos es el siguiente:¿Podemos
escuchar la forma de de un tambor?, esto quiere decir: ¿Es posible
encontrar una única forma para un tambor basándonos en el sonido que
este emite?, el problema directo es verdaderamente simple, dado que,
conocida la forma, podemos deducir el sonido que emite y su solución
es conocida ya desde hace mucho tiempo, pero para la solución invesa
la respuesta es que no, dado que dos tambores distintos pueden emitir
el mismo sonido, esto quiere decir que si escuchamos un sonido no
podríamos distinguir cual es la forma del que lo omite.

Las teorías físicas nos permiten hacer predicciones: Dada una
descripción completa de un sistema físico nosotros podemos predecir el
resultado de algunos parámetros. El problema de predecir el resultado
de las medidas es llamado \textit{problema de modelación o de
  modelamiento}. El \textit{problema inverso} consiste en que, usando
los resultados actuales inferir los valores de los parámetros que
caracterizan el sistema.

Muchos sistemas físicos pueden ser descritos usando un modelo lineal
de la forma:

$$AX = Y$$ en donde $X$ e $Y$ se consideran espacios de Hilbert.

%todo: dibujo de caja para la causa y el efecto

\subsection{Regularización}

Método de regularización de Tichonoff

La idea básica de la regularización de Tichonoff está relacionada con
la minimización del funcional cuadrático:

$$\Phi_\mu(f,g) = \| Af-g\|^2 + \mu^2 \| f\|^2$$

El minimizador se puede obtener de forma analítica de la siguiente
manera.

$$X = \left(A^*A+\mu^2 I \right)A^*Y$$

alguna bibliografía definen $\mu = \sigma_Y/\sigma_X$ como el
parámetro óptimo.



\subsection{Regularización de Tichonoff paralelo  para problemas de optimización no lineal}

La regularizacion de Tijonov se usa habitualmente para obtener soluciones relativamente suaves, dado que penaliza la norma de la solucion lo que penaliza las gradientes grandes.

Para buscar el óptimo para el siguiente problema de optimización con
restricciones

$f(x)=0$ con $g_i(x) < 0 y h_i=0$ Al generar un sistema no lineal
iterativamente se converge a la solución.

Para resolver un problema no lineal, pero que al mismo tiempo conserve cierta suavidad en la solucion se usara la siguiente tecnica.


