
\section{Introducción}

% informacion adicional, restricciones
% informacion de la producción
% busqueda de un volumen

% el modelo matemát


ico implica un aumento en la matrz por invertir en funcion
% de la cantidad de restricciones y cantidad de nodos con que se quere estimar el asunto.
% el problema de discretización implica agregar un error adicional

% un problema computacional
% estructura de datos, hardware.
% que es la gravimetría

El concepto de Problema inverso, como su nombre lo indica, el camino contrario al problema directo
Todo efecto proviene de una causa, ( ... ) un problema se puede denominar como directo si conocemos las leyes que rigen un fenómeno y las causa de un fenómeno con lo cual es muy directo computar algunos parámetros que caracterizan a los efectos. El problema inverso es distinto está inscritos en una rama relativamente reciente de las matemáticas.  En el problema inverso se intenta deducir las causas (resumidas en los parámetros de un modelo) a partir de los datos observados. Dado que para estimar los parámetros del modelo, nos basamos en métodos que dependen de los datos observados hace que nos enfrentemos a un conjunto de preguntas que son de importancia resolver y que implican respuestas en una formulación matemática, lo que implicaría la posibilidad de un diseño de computo suficiente para obtener.

Además, para tratar computacionalmente el problema, los modelos deben ser discretizados y esto introduce errores adicionales.




La gravimetría es un proceso importante en la búsqueda de depósitos minerales 

El Hace referencia a la computación de alto rendimiento para resolver un
problema físico que nace fundamentalmente de la minería.

El presente trabajo presenta algunos avances que hacen referencia a la
reconsideración de dominios más generales lo que impacta directamente
en los métodos computacionales y abre camino para el uso más eficiente
del hardware mediante computación paralela.

De esto se deduce que el modelo posiblemente mejorable en los siguientes aspectos: 
\begin{enumerate}
\item Considerar el volumen como una nueva restricción.
\item Mejorar la velocidad de convergencia del método numérico.
\item Aspectos de visualización de los resultados.
\item considerar distribuciones más generales para la estimación de la
  forma del volumen.
\item Reestimar el parametro $\alpha$ para la regularización de
  Tychonoff para el problema mal puesto.
\end{enumerate}


\subsection{objetivos generales}
Exponer los métodos computacionlaes para resolver problemas científicos que
requieren de alto rendimiento.

Formular un método general para estimar volúmenes que generen campos
escalares cocidos. Crear un nuevo modelo computacional de problema
inverso para poder reconstruir el volumen buscado.

Comparar los resultados anteriores ...


\subsection{objetivos específicos}

crear una aplicaciones que haga uso de un cluster de alto desempeño mediante
computación paralela me paso de mensajes para la resolución de un problema de
gravimetría.

Por otra parte, exportar librerías de mallado que ya se encuntran en
C++ a python, como lo son TetGen, meiante swig y específicamente
mediante una biblioteca llamada instant, esta parte se hará al final
del desarrollo ya que pertenece a una parte de optimización de lo ya
existente.

%visualizacion de los objetivos mediante el siguiente diagrama

