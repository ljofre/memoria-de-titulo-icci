
\section{Ejemplo bidimensional}

Consideremos la curva cerrada simple $C$ en el plano XY que encierra a
un área $\Omega$ y $F(x,y)=\frac{y}{\left(x^2+y^2\right)^{\frac{3}{2}}}$ un campo escalar
genereado por una unidad diferencial $dA$ entonces el campo escalar en el
punto $(x,y)$ generado por la totalidad del área encerrada por la
curva cerrada simple $C$ es
$$\psi(x,y) = \int_\Omega F(u-x,v-y)dudv$$

$$\psi(x,y) = \int_\Omega \frac{1}{(x-u)^2+(y-v)^2}dudv$$


Aplicando el teorema de Green
$$\psi(x,y) = \int_{\partial \Omega} Pdu+Qdv$$

en donde se cumple que
$\frac{\partial Q}{\partial u}-\frac{\partial P}{\partial
  v}=\frac{1}{(x-u)^2+(y-v)^2}$. Si
  consideramos que $P$ es constante en términos de $v$ entonces
  $\frac{\partial P}{\partial v}=0$, luego:

$$\frac{\partial Q}{\partial u}=\frac{1}{(u-x)^2+(v-y)^2}$$
luego $Q(u,v)=\int \frac{1}{(u-x)^2+(v-y)^2}du+h(v) $

Como $C$ es una curva cerrada simple, entonces $\int h(v)dv=0$,
finalmente, el campo escalar en todo queda definido en todo el espacio
como la siguiente función.

$$\psi(x,y)=\oint_{\partial \Omega} Q(u-x,v-y)dv$$

Si discretizamos la integral
$$\tilde \psi(x,y)=\sum_i Q(u_i-x,v_i-y)(v_{i}-v_{i-1})$$

	finalmente, se desean buscar nodos \nodes  que minimice el
error cuadrático

$$I(\Omega) = \sum_k ( \psi_k -\tilde \psi(x_k,y_k))^2$$
Se puede demostrar que es equivalente a buscar los nodos 