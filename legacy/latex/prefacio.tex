
\section{Prefacio}

% diagrama de la idea:
% (1) el concepto errado de computación cuando se confunde con la informática
% (2) la computación como técnica al servicio de las matemáticas, las matemáticas al servicio de la ingeniería.
% (3) La computación tiene sus inicios en el cálculo numérico.
% (4) Algebra lineal herramienta del computo para resolver problemas de algebra lineal


%terminar esta idea
Esto ha resultado en que resolver problemas de la física computacionalmente si es que nos
podemos referir a la computación solo como cálculo, es
una estrategia que ha ido tomando fuerza dada la
posibilidad de obtener soluciones cada vez mas eficientemente a problemas cada vez más grandes. Esto se
produce por evolución de la capacidad de computo de los procesadores,
y estudio de las estructuras de datos 
solución en tiempos de cómputo cada vez de menor orden en espacio para computo y tiempo de cálculo, también métodos numéricos resuelven problemas clásicos del álgebra lineal que aparecen al discretizar sistemas continuos que no se pueden resolver analiticamente.

%Acerca de los algoritmos y el hardware




%acerca de el poder de cómputo

%explicar que el error conceptual generado
Esto de la computación, que se refiere a la capacidad a la capacidad de uso
intensivo de computadores, concepto ligeramente impromipio, dado que
el instrumento no define la técnica 

%ingresar este texto
estudiando modelos algorítmicos de 
fenómenos físicos, permite resolver problemas que no son tratables 
analíticamente, o como se suele referir, requieren de métodos numérico.

Luego, la computación permite un interesante juego teórico: Si se tiene un
modelo detallado de la realidad que se aspira a comprender, se puede puede
simular el comportamiento en la memoria de un computador para lograr medir
cantidades relevantes, y en la medida de que la computación y
algoritmos avanzan nos permiten generar herramientas automatizadas.


El presente trabajo tiene nace en el contexto de crear las
herramientas necesarias para la resolución más general de una linea de
investigación de gravimetría para el laboratorio de modelamiento
matemático para geomecánica (MMGEO) que tenía como objetivo estimar el
cave-back a partir de las mediciones de gravedad (verticales) en
terreno. Dada las necesidades de considerar geometrías más generales y
además considerar la información de la producción de material de la
mina es necesario hacer consideración de una nuevo método numérico que
se base en otros supuestos %cuales supuestos

La forma en que se desarrolló este trabajo depende fuertemente de la
evolución de cada uno de los problemas puntuales que he tenido la
explorados, buscando metodologías para la resolución de problemas,
sirviendo de apoyo para otros matemáticos o creando nuevas
implementaciones para los papers que instuiamos en que se podrán
encontrar las soluciones para los subproblemas.

Esto quiere decir que el desarrollo, aunque consideraba de mucho
material y ya mucha implementación para la fecha en que se inicio esta
tesis, gran parte del trabajo fue darle una secuencialidad, ordenando,
revisando y desechando los multiples borradores en papel y papers
sueltos.

Mucho tiene de resumen del trabajo ya hecho, pero así mismo ese
resumen recoje los argumentos necesarios para justificar los futuros
trabajos que pueden ser posibles o en los que ya se están trabajando.

Es posible que cada uno de los capítulos de este trabajo tiendan a
parecer un trabajo independiente, eso es totalmente intencional, las
razones son que cada uno de los términos de este trabajo se han
desarrollado de forma independiente pero sin una conexión entre estas,
cada una ha intentado trabajar con la física de forma
independiente. Esto más que un defecto puede considerarse una
oportunidad, dado que, un trabajo importante seráa la interelación de
distintos modelos para obtener las generalizaciones de un modelo
unificado.


