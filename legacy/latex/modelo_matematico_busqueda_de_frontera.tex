
% planteamiento del problema en 2d

Los métodos de gravimetría considerando el campo gravitacional en las
coordenadas espaciales, desde un principio se sabe que estas mediciones son muy
inestables dado lo débil del campo gravitacional. Se puede intuir que el campo
gravitacional de la tierra se asume constante en la superficie de la tierra
considerando sus variaciones como despreciables. Mas vano si ignorar estas
variaciones al considerar la existencia de los instrumentos adecuados con un
nivel de precisión que permite medir %aquí no se
que ofrecen una precisión que permiten deducir las propiedades de la masa que
genera el campo gravitacional en todo el espacio.

En el presente capítulo 

% planteamiento de la aplicación del teorema de Green

caso 2d

Sea $f:[a,b]\to \mathbb{R}^+ \cup 0$ y $\Omega$ un %bo teb  

$f(x) = \int_\Omega g(x-u,v)dudv$

% repasando sobre lo ya escrito
esta frontera no es fija en el tiempo, dado que las mediciones de gravedad van
cambiando en el tiempo, implica que estas superficies están sujetas a
actualización en el tiempo. El método científico nos orienta a refinar
incrementalmente el modelo de tal manera de recoger las interpretaciones físicas
como argumentación para el algoritmo

% discretización y planteamiento del sistema no lineal

% agregar las restricciones de area del dominio buscado

% Considerar varias superficies simultaneamente