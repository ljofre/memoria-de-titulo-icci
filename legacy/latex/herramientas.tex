\section{Herramientas computacionales}
Para la resolución del problema se usarán las siguientes herramientas.

\subsection{tetgen}

cómo en su página oficial reza

TetGen es una programa y librería para general dominios tetrahedrales
de mallas para dominios poliédricos. TetGen genera TetGen is a program
to generate tetrahedral meshes of any 3D polyhedral domains. TetGen
generates exact constrained Delaunay tetrahedralizations, boundary
conforming Delaunay meshes, and Voronoi partitions. The following
pictures respectively illustrate a 3D polyhedral domain (left), a
boundary conforming Delaunay tetrahedral mesh (middle), and its dual -
a Voronoi partition (right).
    
TetGen provides various features to generate good quality and adaptive
tetrahedral meshes suitable for numerical methods, such as finite
element or finite volume methods. For more information of TetGen,
please take a look at a list of features.  TetGen is written in
C++. It can be compiled into either a standalone program invoked from
command-line or a library for linking with other programs. All major
operating systems, e.g. Unix/Linux, MacOS, Windows, etc, are supported

\subsection{boost.Python}

Como el lenguaje principal con que se va a trabajar en python pero las
funcionalidades deben tener un performance superior, se envolverán las
funcionalidades desde C++ y se expondran a python.

Is a C++ library which enables seamless interoperability between C++
and the Python programming language. The new version has been
rewritten from the ground up, with a more convenient and flexible
interface, and many new capabilities, including support for:
References and Pointers Globally Registered Type Coercions Automatic
Cross-Module Type Conversions Efficient Function Overloading C++ to
Python Exception Translation Default Arguments Keyword Arguments
Manipulating Python objects in C++ Exporting C++ Iterators as Python
Iterators Documentation Strings The development of these features was
funded in part by grants to Boost Consulting from the Lawrence
Livermore National Laboratories and by the Computational
Crystallography Initiative at Lawrence Berkeley National Laboratories.

\section{Swig}

Otra alternativa para pasar las librerías de alto desempeño a python

SWIG is a software development tool that connects programs written in
C and C++ with a variety of high-level programming languages. SWIG is
used with different types of target languages including common
scripting languages such as Javascript, Perl, PHP, Python, Tcl and
Ruby. The list of supported languages also includes non-scripting
languages such as C-Sharp, Common Lisp (CLISP, Allegro CL, CFFI, UFFI), D,
Go language, Java including Android, Lua, Modula-3, OCAML, Octave and
R. Also several interpreted and compiled Scheme implementations
(Guile, MzScheme/Racket, Chicken) are supported. SWIG is most commonly
used to create high-level interpreted or compiled programming
environments, user interfaces, and as a tool for testing and
prototyping C/C++ software. SWIG is typically used to parse C/C++
interfaces and generate the 'glue code' required for the above target
languages to call into the C/C++ code. SWIG can also export its parse
tree in the form of XML and Lisp s-expressions. SWIG is free software
and the code that SWIG generates is compatible with both commercial
and non-commercial projects.