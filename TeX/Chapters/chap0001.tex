
\chapter{Antecedentes generales}



\section{Introducción}

Una de las tareas más importantes en un observatorio sismológico es la
localización de fuentes sísmicas. Esto significa determinar tanto las
coordenadas del hipocentro como el tiempo de origen de estos eventos. Puesto
que el volumen de mediciones sísmicas es alta implican una alta demanda
computacional para poder aprovechar de toda la información generada por los
sismogramas.
Se presenta entonces el siguiente trabajo de título con un doble objetivo: Por
un lado el diseño  de un algorítmico para la localización de hipocentros
sísmico, tiempo de origen y caracterización las fuentes sísmicas. Por otra parte es la
automatización de este proceso de estimación mediante una librería que puede ser
empleada en una aplicación o servicio computacionales.
Estos modelos y algoritmos fueron aplicados en el contexto de la minería
subterránea a partir de las mediciones sísmicas disponibles. Los datos con
que se testeó el modelo y la implementación fueron obtenidos de las mediciones
de sismicidad de distintas minas de explotación subterránea de la división el
teniente, Codelco.
Se dividió este trabajo en los siguientes capítulos:
Capítulo I (Planteamiento del problema): La teoría sísmica y la interpretación
de las mediciones sismográficas suficiente para la comprensión de la problemática.
Capítulo II (Marco teórico) Se analizó desde la teoría de la elasticidad cuyas
ecuaciones nos permite modelar los fenómenos sísmicos para un medio elástico, homogéneo e
infinito y permite deducir la función de Green la cual es esencial para el
diseño del algoritmo planteado en este trabajo.
Capítulo IIb: A partir de los datos de entrenamiento se diseña una red neuronal
que estima el tiempo de ocurrencia de la onda $P$ y la onda $S$.
Capítulo III: Se describe el modelos numérico para resolver el problema, basado
en los resultados expuestos en el capítulo II, este capítulo describe el modelo
y algoritmo matemático que permitirá la estimación del hipocentro sísmico usando
toda la información de la señal de los sismogramas, esto es una innovación con
respecto los actuales algoritmos que hacen una estimación del hipocentro
mediante el tiempo de arribo de la onda p y s, aunque usa esta estimación como
condición inicial. En esta parte de describirá el concepto de problema inverso y
una técnica para aplicarlo al problema elástico. Se justificara la demanda
computacional que involucra.
Capítulo IV: Se describirá el algoritmo y estructura de datos que se seleccionó
para resolver el problema y su versión computacional.
Capítulo V: está reservado para la descripción la versión computacional actual
de la solución, o sea, la implementación del método, la arquitectura de software
que se idónea que  según actuales autores se enmarca en el área de Big Data y
Ciencia de los Datos.
Capítulo VI: se describe la calidad del modelo y sus futuras aplicaciones,
posibles mejoras y recomendaciones.

Anexo A están los desarrollos matemáticos para los modelos físicos considerados
en este trabajo. En el Anexo B están las pruebas para sismos sintéticos.

\section{Objetivos}

\subsection{General}

\subsection{Específico}

\subsection{Justificación del tema}
\subsubsection{Motivación}

\subsection{Alcances y limitaciones}

\subsection{Metodología y plan de trabajo}

\subsection{Resultados esperados}

