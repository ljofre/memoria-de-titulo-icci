\chapter{Marco Teórico}

En este capítulo se describirá la naturaleza de esas ondas elásticas y el
análisis de los sismogramas.

\section{Elementos de sismología}

La tierra está compuesta de silicatos y aleaciones de hierro con las que
adquiere propiedades que sobre la amplia condiciones existentes de temperatura y
presión en el planeta, el material responde de manera elástica.
Por otro lado bajo la aplicación de fuerzas de baja magnitud y se comporta de
manera viscosa y plástica sometida a fuerzas de larga duración. (cita requerida)

Las vibraciones mecánicas resultantes de este comportamiento cuasi-elástico,
implican la excitación y propagación de ondas elásticas en el interior. Muchas
de estas ondas que se irradian son imperceptibles pero se pueden detectar con
modernos instrumentos \cite[pp. 1]{Shearer2009}.

La sismología ocupa una interesante posición en el campo de la geofísica y
ciencias de la tierra. Presenta interesantes problemas teóricos que involucran
el análisis de ondas elásticas que se propagan en medios complejos \cite{Shearer2009}.

Mucha de la física subyacente no es más avanzada que la segunda ley de Newton
($F=ma$), pero la complicación es introducir fuentes sísmicas realístas y
estructuras que motivan un tratamiento matemático sofisticado y el uso de
poderosos computadores \cite{Shearer2009}.



de estas ondas Esas ondas son un desplazamiento físico como movimiento telúrico
y son capturados por instrumentos  llamados sismómetros los cuales son
almacenados para su  análisis científico. (cita requerida)

La sismología es el estudio de la generación, propagación, y registro de las
ondas elásticas de la tierra, y de las fuentes que la causan. 

Ambas, los sismos naturales y los producidos por el hombre pueden generar ondas
sísmicas:
perturbaciones elásticas que se expanden esféricamente hacia fuera desde la
fuente como resultado de un desequilibrio transitorio de las tensiones de la
roca. Las propiedades de las ondas sísmicas están gobernadas por las leyes de la
física de solidos elásticos, la cual está completamente descrita en la teoría de
la elasto-dinámica. (cita requerida)

%%%%%%%%%%%%%%%%%%%%%%%%%%%%%%%%%%%%%%%%%%%%%%%%%%%%%%%%%%%%%%%%%%%%%%%%%%%%%%%%
Este conjunto de teorías basado en la mecánica de cuerpos continuos, elasticidad
lineal y matemáticas aplicadas datan de principio del 1800, proveyendo de un
marco de trabajo para el análisis de las ondas elásticas en la Tierra.

A mediados de 1800 la teoría de ondas elásticas empezó a ser desarrollada por
Cauchy, Poisson, Stokes, Rayleigh, y otros \cite[pp. 2]{Shearer2009}

%%%%%%%%%%%%%%%%%%%%%%%%%%%%%%%%%%%%%%%%%%%%%%%%%%%%%%%%%%%%%%%%%%%%%%%%%%%%%%%%

La geofísica surgió como necesidad de justificar los fenómenos observados por la
geología [LocEpic 2011]

La sismología es la rama de la física de estudia los terremotos y los fenómenos
asociados a ellos. En sus inicios era una ciencia observacional hasta que en el
último siglo su desarrollo fue avanzando por el desarrollo tecnológico
alcanzado. Este desarrollo se ve hoy en la exploración sísmica del petróleo, la
evaluación de riesgo sísmico, la planificación del uso del suelo. [Localización
de epicentros 2011]

Una rango amplio de escalas de sismos son considerados en la sismología, para
ambos los variados tipos de fuentes y los diversos tipos de ondas que se
producen por estos. Los más pequeños de estos microsismos de los detectables
tienen un momento sísmico (una importante cantidad física que es igual al
producto entre el área de la superficie de falla, la rigidez de la roca, y el
desplazamiento medio en la falla) tienen un orden de $10^5 Nm$, y en los grandes
terremotos tienen un momento de la magnitud de $10^23 Nm$. La amplitud de las
ondas sísmicas son directamente proporcionales al momento sísmico; Así los
desplazamientos de las ondas sísmicas abarcan un enorme rango. Las ondas
sísmicas que usualmente son usadas en exploración sismológica tienen frecuencias
astas alrededor de los 200 Hz, mientras las de un largo periodo de los grandes
terremotos tienen una frecuencia de $3x10^{-4} Hz$. Así, los movimientos de la
tierra que abarcan un rango de frecuencias de $10^7$ son interesantes.

De hecho el estudio de las fuentes sísmicas está más allá de extender el rango
de interés a la frecuencia nula, o deformaciones estáticas, fallas o explosiones
cercanas …

Uno de los mayores desafíos plantados por el amplio ancho de banda de
frecuencias, y el rango de amplitudes


Sismogramas

Los sismómetros miden la velocidad de una onda que se propaga por el medio en
tres dimensiones, dos componentes horizontales y uno vertical. Cuando ocurre un
evento sísimico, diferentes fases sísmicas llegan a diferentes distancias del
hipocentro, y en diferentes componentes del instrumento. Las diferentes fases
sísmicas tienen un diferente rango de frecuencias. Ondas de cuerpo en general
tienen mayor frecuencia que las ondas de superficie.

Los sismogramas son “… instrumentos que graban de las vibraciones mecánicas de
la tierra” [Modern Global Seismology]

Ondas sísmicas

Fuentes sísmicas

Localización de epicentros

Una de las tareas más importantes en un observatorio sismológico es la
localización de fuentes sísmicas lo que significa determinar la localización del
hipocentro y el tiempo de origen, para ello actualmente existen variadas
técnicas y aún es un activo campo de investigación. En general, para determinar
la ubicación del sismo se requiere la identificación de fases sísmicas y las
mediciones de tiempo de llegada de las ondas principales, así como el
conocimiento de algunos parámetros físicos, como lo son la densidad del medio y
la velocidad de propagación de las ondas entre el hipocentro y la estación
sísmica. Luego de estimada la ubicación se puede estimar el tiempo de origen.

Localización con una sola estación Localización con varias estaciones



% Manual de Prevención Sísmica – INPRES – 1995 Earthquakes and Geological
% Discovery –
% 
% 
% Kulhánek - IASPEI/UNESCO - 1990 Introducción a la Geofísica – Benjamín F.
% Howell, Jr. – Ediciones Omega - 1962


\section{Teoría de la elasticidad y ondas sísmicas}

La sismología envuelve el análisis del movimiento de la tierra producida por la
energía de fuentes dentro de la tierra, tal como terremotos sobre fallas o
explosiones. Excepto en el vecindad inmediata de la fuente, la mayoría del
desplazamiento es efímero; El medio vuelve a su posición inicial después  de que
el movimiento transmitido a declinado. Vibraciones de este tipo involucran
pequeñas deformaciones elásticas, o tensiones. La teoría de la elasticidad
provee las relaciones matemáticas entre la tensión y la presiones en el medio, y
esto ha generado una vasta literatura llena de teoría y documentación empírica
del comportamiento elástico del medio terrestre. En este texto se desarrolla
solo lo básico de la teoría de la elasticidad requerida para aplicaciones
sismológicas, incluido el concepto de presión y tensión, las ecuaciones de
equilibro y movimiento, y la solución fundamental de la ecuación de movimiento:
Las ondas sísmicas.

El desarrollo que se hará de la teoría de la elasticidad sigue lo típico en los
textos de mecánica de sólidos orientado a demostrar la existencia de la función
de Green de la onda elástica que es parte fundamental del algoritmo propuesto.

En el estudio de sólidos, un útil concepto teórico que trata con el fenómeno
macroscópico es el de medio continuo, en el cual la materia se distribuye
continuamente en el medio. Dentro de este material continuo podemos definir
funciones matemáticas para el campo de desplazamiento, tensiones y presión, la
cuales tienen derivadas espaciales continuas. Aplicando simples leyes de la
física al medio continuo, permite explicar casi toda la información de los
sismogramas.

La Sismología en su mayor parte, en la práctica fenómenos con muy poca
deformación (el cambio relativo es del orden de $10^6$) sobre muy cortos
periodos de tiempo ($<3600$ s). Esto simplifica mucho los modelos elásticos ya
que se pueden omitir algún tipo de fase plástica del medio, se analiza solo como
un problema elástico basado en la teoría infinitesimal de tensiones. En la
vecindad más inmediata de la fuente sísmica, o cuando se considera una
deformación de gran escala deformaciones de fallas (como en la geología
estructural), se requiere de una teoría de las tensiones más general.

La relación entre fuerzas y deformaciones en la teoría de tensiones
infinitesimal está validada empíricamente basada en las ley constitutiva o Ley
de Hook. La deformación es en función de las propiedades del material, como lo
son la densidad, la rigidez (resistencia a ser cortado) y la incompresibilidad
(resistencia a cambiar de volumen). Las propiedades del material son conocidos
como módulos elásticos. Cuando la tensión varia en el tiempo, la deformación
varia similarmente, y el balance entre la deformación y la tensión resulta en
ondas sísmicas. Esas ondas viajan en una velocidad que depende de los módulos
elásticos del medio y son gobernadas por la ecuación de movimiento. Las ondas
símicas son el desplazamiento de las partículas la cual se transforma en cada
vez más compleja como la onda que se expande a través del cuerpo sólido. Se
procederá a mostrar como esas ondas aparecen y como se representan
matemáticamente.


Deformación

Como la sismología está directamente asociada a la medida del movimiento en un
medio, se empezará por considerar cómo describir el movimiento dentro de un
sólido es descrito. Emplearemos enteonces una descripción Lagrangiana, en cual
el desaplzamiento de una partícula es descrita comoo una función del tiempo y el
espacio. Este es el sistema natural en sismología ya que es la forma en que los
sismogramas registran el desplazamiento en un punto en función del tiempo. Sea
en una continua distribución de partículas, así u(x,t) describe el campo de
desplazamiento de cada punto en el medio, donde se está en libertar de elegir un
sistema conveniente de referencia.

Un medio puede someterse a  dos tipos de movimiento, (1) todo el sufre una
translación o rotación (2) o sufre deformaciones, La translación y rotación
puede ser descrita con un simple vector común para todos los puntos del medio.
Pero, para un problema sísmico lo que importa son las deformaciones internas del
medio, la cual intrinsicamente implica una variación espacial y temporal del
medio $u(x,t)$

Las deformacione en el medio están formadas por las que envuelven los cambios de
logintud y distorciones angulares. Considere un cuerpo que está inicialmente no
deformado y en dos puntos externos O y P no hay cargas. Estos puntos están
conectados por la línea Delta s, Cundo fuerzas son aplicadas sobre el cuerpo, la
deformación nueve O y P  a O’ y P’, resectivamente, la línea cual conecta estos
puntos ahora es delta s prima. Para describir la deformación del medio, se debe
caracterizar el cambio de distancia entre los dos puntos y cualquier rotación de
la línea delta s’ relativa al material que la rodea.

Para eso vamos a introducir el concepto de gradiente espacial del campo de
desplazamiento, o deformación. La deformación axial como medida de elongación,
definida como
La deformación axial implica el cambio relativo de la distancia entre los
puntos. El segmento de línea 0’P’ podría no haber sufrido cambios en su logitud,
pero podría haber rotado respecto con el material que la envuelve. Su
consideramos una línea perpendicular al segmento OQ en el medio sin
deformaciones que se mueve al punto O’Q’,


\section{Relación entre la tensión y el desplazamiento}

\section{Ecuación de movimiento}

\section{Función de Green para la ecuación diferencial elástica}

Asumiento que la propagación de la onda sismica en el medio homogéneo con
dominio infinito está bien modelada. Es decir
$\mathbf{u}\left(\mathbf{x},t\right)$ representa el desplazamiento en el punto
$\mathbf{x}$ en el tiempo $t$ debido a la onda, entonces asumimos que la
propagación de la onda $u$ está modelada adecuadamente por la ecuación.

$$\rho\ddot{\mathbf{u}}=\mathbf{f}+\left(\lambda+2\mu\right)\nabla\left(\nabla\cdot\mathbf{u}\right)-\mu\nabla\times\left(\nabla\times\mathbf{u}\right)$$
 

donde $\rho$ es la densidad del medio, $\lambda$, $\mu$ son los parámtros de
Lamme que se asumen constantes y $\mathbf{f}$ es la fuente sismica expresada
como una fuerza,

Para resolver la ecuación existe la expresión explícita de la función de Green,
es decir del campo de desplazamiento en la dirección $i$ debido a un impulso en
la dirección $j$ en el tiempo $t=0$ .

\begin{equation}
\begin{split}
G_{ij}\left(\mathbf{x},t\right)
=
\frac{1}{4\pi\rho}\left(3\gamma_{i}\gamma_{j}-\delta_{ij}\right)\frac{t}{r^{3}}1_{\left[\frac{r}{\alpha},\frac{r}{\beta}\right]}\left(t\right)
\\
&
+\frac{1}{4\pi\rho\alpha^{2}}\gamma_{i}\gamma_{j}\frac{1}{r}\delta\left(t-\frac{r}{\alpha}\right)
\\&
-\frac{1}{4\pi\rho\beta^{2}}\left(\gamma_{i}\gamma_{j}-\delta_{ij}\right)\frac{1}{r}\delta\left(t-\frac{r}{\beta}\right)
\end{split}
\end{equation} 

\section{Estimación de la fuente utilizando mínimos cuadrados}

Como describimos anteriormente, una fuente sismica carcterizada por una fuerza
localizada en $\mbox{x}_{0}$ modulada temporalmente según
$\mbox{s}\left(t\right)$ , se puede incluir en como un lado
derecho de la forma
$\mbox{s}\left(t\right)\delta_{\mathbf{x}_{0}}\left(\mbox{x}\right)$. Denotamos
la función de Green como $\mbox{G}\left(\mathbf{x}|t\right)$ , podemos calcular
el campo de desplazamiento de la onda en el punto $\mathbf{x}$ como en el tiempo
$t$, como
$u\left(x,t\right)=\mbox{G}\left(\mbox{\textbf{x}}-\mathbf{x}_{0}|t\right)\ast\mathbf{s}\left(t\right)$
 

donde las mediciones corresponden al campo de desplazamiento, a lo largo del
tiempo, en las posiciones $\left\{ r_{k}\right\} _{k=1}^{K}$. Es decir, nuestras
mediciones corresponden al conocimiento de
$u_{k}\left(t\right) = u\left(\mathbf{r}_{k},t\right)$ para
$t \in T_{k}\subset\mathbb{R}^{+}$.

Asumiendo qe la posición $\mathbf{x}_{0}$ de la fuente es conocida y que
$\left(t\right)$ está soportado cerca de un tiempo $t_{0}$ conocido, queremos
plantear un metodo de reconstrucción de $\left(t\right)$ mediante minimos cuadrados.

Consideremos $\left\{ \varphi_{j}\right\}_{j=1}^{J}$ una familia de funciones
linealmente independientes, soportadas alrededor de $t_{0}$ , que nos permitan
describir $\left(t\right)$ aproximadamente. Es decir, consideremos que
$\left(t\right)=\sum_{j=1}^{J}\alpha_{j}\varphi_{j}\left(t\right)$
 

La familia $\left\{ \varphi_{j}\right\}_{j=1}^{J}$
  está dada y el objetivo es encontrar $\left\{ \alpha_{j}\right\}_{j=1}^{J}$
  a partir de las mediciones. Para esto según el modelo de proparación de onda,
  tenemos que $\alpha_{j}$ debería cumplir, para todo $k=1,\ldots,K$
  y para todo $t\in T_{k}$
 , que

\begin{eqnarray}
u_{k}\left(t\right)	=	\left[G\left(r_{k}-\mathbf{x}_{0}\right)\ast
s\right]\left(t\right) \\
 u_{k}\left(t\right)=\left[G\left(r_{k}-\mathbf{x}_{0}\right)\ast\sum_{j=1}^{J}\alpha_{j}\varphi_{j}\right]\left(t\right)
 \\
 u_{k}\left(t\right)	=
 \sum_{j=1}^{J}\alpha_{j}\left[G\left(r_{k}-\mathbf{x}_{0}\right)\ast\varphi_{j}\right]\left(t\right)
\end{eqnarray}

 

Encontrar los valores de $\left\{ \alpha_{j}\right\} _{j=1}^{J}$
 resolviendo la ecuación anterior, o escogiendo aquellos que se acerquen lo más
posible a resolverla. Más explícitamente, encontrar $\alpha_{j}$ como

$$\left\{ \alpha_{j}\right\} =\arg\min_{\left\{ \alpha_{j}\right\}
}\left(\sum_{k=1}^{K}\int_{T_{k}}\left(u_{k}\left(t\right)-\sum_{j=1}^{J}\alpha_{j}\left[G\left(r_{k}-\mathbf{x}_{0}\right)\ast\varphi_{j}\right]\left(t\right)\right)^{2}dt\right)$$
 

Como las mediciones temporales son efectivamente discretas, lo anterior se puede escribir de la manera matricial como 

$$\hat{\alpha}=\arg\min_{\alpha}\left\Vert u-\alpha\cdot A\right\Vert ^{2}$$
 

donde $u=\left(u_{k}\left(t\right)\right)_{k=\text{1,\ensuremath{\ldots},K}}$
 

$\alpha=\left(\alpha_{1},\ldots,\alpha_{J}\right)$
 

y
$$A=\left(\left[G\left(r_{k}-\mathbf{x}_{0}\right)\ast\varphi_{j}\right]\left(t\right)\right)_{k=1,\ldots,K;t\in
T_{k};j=1,\ldots,J}$$
 

escritos de manera adecuada. El problema anterior es un problema estándar de
minimos cuadrados  y la solución (el de menor norma en caso de admitir multiples
minimizantes)  se escribe como $\hat{\alpha}=u\cdot
A^{t}\cdot\left(AA^{t}\right)^{+}$ donde $B^{+}$ es la pseudo inversa de Moore
Penrose cuando $B$ no es invertible.

Esto entrega una reconstrucción de $\left(t\right)$ como
$\hat{s}\left(t\right)=\sum_{j=1}^{J}\hat{\alpha}_{j}\varphi_{j}\left(t\right)$


\section{Ecuación de onda: La onda P y S}

El 1900 Richard Oldham reporta la identificación de las ondas P, S, y ondas
superficiales en un sismograma. 

La onda S es una onda de mayor amplitud pero de velocidad de propagación más
lenta, en cambio la onda P es del menor amplitud pero de mayor velocidad de
propagación. Al cumplirse esta propiedad los sismogramas que están más lejos del
hipocentro perciben ambas ondas en una diferencia de tiempo mayor que los que
están más cercanos al mismo.

Las ondas P son aquellas que se transmiten por la compresión y dilatación de las
partículas. Ejemplo de estas ondas de compresión es el sonido (ondas acústicas),
las partículas se mueven en el mismo sentido en que la onda se propaga (figura
1.1-1) y son las que tienen mayor velocidad al desplazarse. Para un material
elástico e isotrópico el modelo matemático para la velocidad es \cite{stein2009introduction}

$$\alpha = \sqrt{\frac{\lambda + 2\mu}{\rho}}$$

Donde $\alpha$ es la velocidad de ondas $P$, $\lambda$ es la constante de Lamé,
$\mu$ es el módulo de rigidez y $\rho$ es la densidad, este tipo de onda puede
propagarse tanto en líquidos como en sólidos ya que no involucra un cambio en el volumen.

Por otro lado, las ondas S se transmiten perpendicularmente a la dirección de
propagación de la onda, por lo que tienen una componente vertical (SV) y una
horizontal (SH). Cuando solo se desplazan en una dirección se dice que es una
onda polarizada. Se les conocen como ondas de cizalla o de corte. La velocidad
de propagación es menor a la de las ondas P (figura 1.1-2) siendo su modelo
matemático el siguiente \cite{stein2009introduction}:

$$\beta = \sqrt{\frac{\mu}{\rho}}$$





